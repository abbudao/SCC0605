\documentclass[a4paper,12pt]{article}
\usepackage[top=2.5cm, bottom=2.5cm, left=3cm, right=3cm]{geometry} %for margin
\usepackage[utf8]{inputenc}
\usepackage{xcolor}
\usepackage[portuguese]{babel}
\usepackage[most]{tcolorbox}
\usepackage[colorlinks=true]{hyperref}
\usepackage{mathtools}
\usepackage{syntax}
\usepackage{cite}
\usepackage{listings}

\usepackage{etoolbox} % for patching
\makeatletter
% define the main command on the model of the original one
% we add stepping the counter and typesetting the number
\def\gr@implnumbereditem<#1> #2 {%
  \stepcounter{grammarline}%
  \sbox\z@{\hskip\labelsep\grammarlabel{#1}{#2}}
  \strut\@@par%
  \vskip-\parskip%
  \vskip-\baselineskip%
  \hrule\@height\z@\@depth\z@\relax%
  \item[%
    \rlap{\hskip\dimexpr\linewidth+\grammarindent\relax %% add the number
          \llap{(\thegrammarline)}}%
    \unhbox\z@]%
  \catcode`\<\active%
}
% copy the grammar environment under a new name
\let\numberedgrammar\grammar
\let\endnumberedgrammar\endgrammar
% now patch the new environment
\patchcmd\numberedgrammar
  {\gr@implitem}
  {\gr@implnumbereditem}
  {}{}
\patchcmd\numberedgrammar
  {\def\alt{\\\llap{\textbar\quad}}}
  {\let\alt\alt@num}
  {}{}

% the command for numbering the \alt lines
\def\alt@num{\\\relax
  \stepcounter{grammarline}%
  \rlap{\hskip\dimexpr\linewidth-\labelwidth+\grammarindent-\labelsep\relax
        \llap{(\thegrammarline)}}% add the number
  \llap{\textbar\quad}}

\newcounter{grammarline}
\makeatother
\newcommand\tab[1][1cm]{\hspace*{#1}}

\usepackage{DocumentStyle}

\title{Compiladores - Entrega 1}

\begin{document}
\begin{titlepage}
  \centering
  \includegraphics[width=0.5\textwidth]{./assets/icmcLogo.pdf}\par\vspace{1cm}
  {\scshape\LARGE Universidade de São Paulo \par}
  \vspace{1cm}
  {\scshape\Large  Teoria da Computação e Compiladores  \par}
  \vspace{1.5cm}
  {\huge\bfseries Trabalho 2 \\ Parser em JASON\par}
  \vspace{2cm}
  {\Large\itshape Felipe Aparecido Garcia 9368751\par}
  {\Large\itshape Marcelo Bertoldi Diani 9313291\par}
  {\Large\itshape Marco Adriano Tette Schaefer 9266302\par}
  {\Large\itshape Pedro Morello Abbud 8058718\par}

  \vfill
  Professora Dra.\par
  \textsc{Sandra Aluisio}

  \vfill

  % Bottom of the page
  {\large \today}
\end{titlepage}

\tableofcontents
\clearpage

\section{Introdução}
	Neste trabalho, foi implementado um \textit{parser} para a linguagem Jason, utilizando a ferramenta JavaCC. Nele, definiram-se as regras da linguagem, com a notação EBNF (\textit{Extended Backus–Naur Form}), e os \textit{tokens} da mesma, a serem gerados pelo \textit{parser}.
    
    Além disso, foram desenvolvidos casos de teste para diversas funcionalidades da linguagem, utilizando o \textit{framework} de teste JUnit para executar todos os testes, apoiado na \emph{Build Tool} chamada \emph{Gradle}. Os testes foram dividos em testes que deveriam ser aceitos pelo \emph{parser} e testes onde o \emph{parser} deveria gerar erros.
    
\section{Questões}
%acho q seria interessante formatar melhor esse texto
	Antes de iniciar a construção do analisador léxico, algumas decisões de projeto devem ser tomadas. Com este propósito, propôs-se perguntas a serem respondidas sobre a linguagem:
    \begin{enumerate}
      \item A linguagem permite comentário dentro de comentário? São sensíveis à linha? Qual o marcador de comentário?
      	
        Resposta: A linguagem não permite comentários dentro de comentário, sendo que o comentários são sensíveis à linha e possuem como marcador o símbolo '\#'. 
        
      \item Os identificadores e as palavras-reservadas serão \textit{case-sensitive}? Ou seja, haverá diferenciação entre letras minúsculas e maiúsculas?
      
      Resposta: Sim, os identificadores e as palavras reservadas serão \textit{case-sensitive}.
      
      \item Os identificadores terão tamanho máximo de caracteres? Se não, há número de caracteres para diferenciação? 
      
      Resposta:Os identificadores não terão um tamanho máximo de caracteres e não haverá número de caracteres para diferenciação.
      
      \item Conjunto de palavras reservadas:
      
            \begin{symbols}[Palavras-Chave da linguagem]
        \begin{align*}\
          \mathtt{V_t} = \big(\lit{begin}, \lit{end}, \lit{program}, \lit{types}, \lit{array}, \lit{record},
          \lit{variables},\\ \lit{real}, \lit{integer}, \lit{string}, \lit{procedure},
          \lit{parameters}, \lit{var}, \\\lit{function}, \lit{returns}, \lit{read}, \lit{set}, \lit{write}, \lit{if}, 					\lit{then},
          \lit{while},\\ \lit{do}, \lit{endwhile}, \lit{untill}, \lit{enduntil}, \lit{call}, \lit{else},
        \lit{endif}, \lit{return}\big)
      \end{align*}
      \end{symbols}
      
      \item  Constantes permitidas: real, integer e string.
      
      \item Caracteres não imprimíveis e símbolos especiais: a linguagem não os aceita.
    \end{enumerate}
\section{Utilzando e construindo a aplicação}
\subsection{Rodando o Parser}
Utilize o arquivo \emph{.jar} fornecido, ou construído nas subseções seguintes, da seguinte forma:
\begin{lstlisting}[language=bash,caption={Executando o Parser}]
#!/bin/bash
java -jar nomeDoJar.jar
\end{lstlisting}
O programa espera como entrada um caminho para um arquivo ou caso não haja argumento de entrada, este rodará iterativamente pelo \emph{STDIN} (\emph{standart input}). Como definido no projeto, caso aconteça algum erro, o parser parará imediatamente e informará qual era \emph{Token} esperada e qual foi a \emph{Token} recebida.

\subsection{Construindo o Parser}
Pode-se construir o \emph{Parser} através dos códigos fontes. Para isto algumas dependências são necessárias:
\begin{itemize}
\item Certifique-se de ter o Java Development Kit 8 ou superior instalado. Caso não o tenha, \href{https://docs.oracle.com/javase/8/docs/technotes/guides/install/install_overview.html}{siga estes passos} ou utilze seu gerenciador de pacotes.
\item Instale o Gradle para automatização da \emph{build}. Encontre os passos para \href{https://gradle.org/install/}{sua distribuição aqui}. Alternativamente, use seu gerenciador de pacotes.
\end{itemize}
Descompacte os arquivos fontes do zip ou clone o repositório:
\begin{lstlisting}[language=bash,caption={Obtendo os arquivos fontes}]
#!/bin/bash
git clone https://github.com/abbudao/SCC0605 
\end{lstlisting}
Na pasta raiz do projeto, construa uma \emph{build}:
\begin{lstlisting}[language=bash,caption={Construção automatizada}]
#!/bin/bash
gradle build
\end{lstlisting}
Os arquivos gerados pelo \emph{JavaCC} estarão no caminho \emph{``build/generated''}
 e o jar automaticamente gerado no caminho \emph{``build/libs''}. Para a construção a suíte de testes já foi rodada e um report detalhado pode ser encontrado em \emph{``build/reports''}.
 
\subsection{Testes}
Para executar-se os testes, o segundo comando pode ser usado:
\begin{lstlisting}[language=bash,caption={Rodando a suíte de testes}]
#!/bin/bash
gradle test 
\end{lstlisting}
O resultado da suite será mostrado no \emph{STDOUT} e um resultado detalhado também pode ser encontrado
em \emph{``build/reports''}.
\end{document}





