\documentclass[a4paper,12pt]{article}
\usepackage[top=2.5cm, bottom=2.5cm, left=3cm, right=3cm]{geometry} %for margin
\usepackage[utf8]{inputenc}
\usepackage{xcolor}
\usepackage[portuguese]{babel}
\usepackage[most]{tcolorbox}
\usepackage[colorlinks=true]{hyperref}
\usepackage{mathtools}
\usepackage{syntax}
\usepackage{cite}
\usepackage{listings}
\usepackage{etoolbox} % for patching
\makeatletter
% define the main command on the model of the original one
% we add stepping the counter and typesetting the number
\def\gr@implnumbereditem<#1> #2 {%
  \stepcounter{grammarline}%
  \sbox\z@{\hskip\labelsep\grammarlabel{#1}{#2}}
  \strut\@@par%
  \vskip-\parskip%
  \vskip-\baselineskip%
  \hrule\@height\z@\@depth\z@\relax%
  \item[%
    \rlap{\hskip\dimexpr\linewidth+\grammarindent\relax %% add the number
          \llap{(\thegrammarline)}}%
    \unhbox\z@]%
  \catcode`\<\active%
}
% copy the grammar environment under a new name
\let\numberedgrammar\grammar
\let\endnumberedgrammar\endgrammar
% now patch the new environment
\patchcmd\numberedgrammar
  {\gr@implitem}
  {\gr@implnumbereditem}
  {}{}
\patchcmd\numberedgrammar
  {\def\alt{\\\llap{\textbar\quad}}}
  {\let\alt\alt@num}
  {}{}

% the command for numbering the \alt lines
\def\alt@num{\\\relax
  \stepcounter{grammarline}%
  \rlap{\hskip\dimexpr\linewidth-\labelwidth+\grammarindent-\labelsep\relax
        \llap{(\thegrammarline)}}% add the number
  \llap{\textbar\quad}}

\newcounter{grammarline}
\makeatother
\newcommand\tab[1][1cm]{\hspace*{#1}}

\usepackage{DocumentStyle}

\title{Compiladores - Entrega 3}

\begin{document}
\begin{titlepage}
  \centering
  \includegraphics[width=0.5\textwidth]{./assets/icmcLogo.pdf}\par\vspace{1cm}
  {\scshape\LARGE Universidade de São Paulo \par}
  \vspace{1cm}
  {\scshape\Large  Teoria da Computação e Compiladores  \par}
  \vspace{1.5cm}
  {\huge\bfseries Trabalho 2 \\ Parser em JASON\par}
  \vspace{2cm}
  {\Large\itshape Felipe Aparecido Garcia 9368751\par}
  {\Large\itshape Marcelo Bertoldi Diani 9313291\par}
  {\Large\itshape Marco Adriano Tette Schaefer 9266302\par}
  {\Large\itshape Pedro Morello Abbud 8058718\par}

  \vfill
  Professora Dra.\par
  \textsc{Sandra Aluisio}

  \vfill

  % Bottom of the page
  {\large \today}
\end{titlepage}

\tableofcontents
\clearpage

\section{Introdução}
	Neste trabalho, desenvolveu-se um analisador semântico para a linguagem Jason com base no analisador sintático feito anteriormente. O analisador criado, recebe como entrada o código analisado pelo léxico e o sintático, e procura por erros semânticos no programa.
    
\section{Considerações Semânticas}
    \begin{enumerate}
    	\item  Tempo de vida, visibilidade e escopo para identificadores
        	
            
            Os identificadores são visíveis apenas no bloco em que foram declarados e em blocos com níveis maiores, e são excluídos ao fim de cada bloco.
            
        
        \item Procedimentos/Funções aninhados
        
        	A linguagem permite procedimentos e funções aninhadas. Para procedimentos,  a regra 24 comprova isso e para funções, a regra 25.
        
        \item Passagem de parâmetros de que tipo?
        	
            Em Jason, os parâmetro podem ser passados ou por valor ou por referência. Segundo a regra 31, pode-se utilizar ou não $"var"$ antes de um parâmetro. Caso este termo opcional seja usado, a passagem será por referência, caso contrário, será por valor.
        
        \item Compilação independente, compilação em separado ou o programa está em um
único arquivo? 
        
        A compilação dos programas é independente.
        
    \end{enumerate}

\section{Definição Formal de Jason}
	Nesta seção encontram-se a definição formal da linguagem, em EBNF, além do conjunto de regras semânticas (RS), assim como os pontos de checagem das RS na linguagem.
    
    %fazer RS e incluir checagens
    \subsection{Regras Semânticas}
    
    \begin{enumerate}
    \item Se declarado(TS,s,nivelcorr) então erro(‘id já declarado’);
insere(TS,s,ref);
    \begin{enumerate}
    	\item Atualiza os campos nível, categoria, valor e tipo\_c através da rotina seta\_atributos com: nivelcorr,
constante, valor obtido da rs(1.b) x sinal ou rs(1.c) x sinal e tipo\_c obtido da rs(1.b) ou rs(1.c).
		\item busca(TS, s, ref, achou);
Se não achou ou (achou e categoria $<>$ constante)
Então erro(‘constante não definida’)
Senão obtém tipo\_c e valor através da rotina obtem\_atributos
		
        \item tipo\_c = inteiro; valor = valor obtido da conversão de s para int 
	\end{enumerate}
	\item Se declarado(TS,s,nivelcorr) então erro(‘id já declarado’);
insere(TS, s, ref);
permissão\_tipo\_array := true;

	\item Atualiza os campos nível, categoria, nbytes e Pont\_tipo\_elementar através da rotina seta\_atributos com:
nívelcorr, tipo, nbytes e tipo elementar com info da rs(10) ou com info do endereço do array quando tipo for
array
	
    \item Se declarado(TS,s,nivelcorr) então erro(‘id já declarado’);
insere(TS,s,ref); permissão\_tipo\_array := true;
	
    \item Atualiza os campos nível, categoria, tipo\_v das variáveis na TS através da rotina seta\_atributos com:
nivelcorr, variável, o endereço fornecido pela rs(10);
    \begin{enumerate}
    \item Salva nível, categoria, tipo\_v
    \end{enumerate}
    
    \item  Se declarado(TS,s,nivelcorr) então erro(‘id já declarado’);
insere(TS,s,ref);
Atualiza os campo nível e categoria através da rotina seta\_atributos com: nivelcorr, procedimento;
npar := 0;
nivelcorr:= nivelcorr + 1;

	\item  Se declarado(TS,s,nivelcorr) então erro(‘id já declarado’);
insere(TS,s,ref);
Atualiza os campos nível e categoria através da rotina seta\_atributos com: nivelcorr, função;
npar := 0;
nivelcorr: nivelcorr + 1;

	\item busca(TS,s,ref,achou);
Se não achou ou (achou e a categoria $<>$ tipo) então erro(‘tipo não definido’)
Senão atualiza tipo\_f;

	\item elimina(TS,nivelcorr);
{Não elimina as informações de tipo e passagem de parâmetros dos parâmetros de procedimentos/funções.
Apagar os identificadores de parâmetros ou encadear a lista de parâmetros (tipo/passagem) na entrada de
procedimentos para posterior checagem de tipos e passagem}
nivelcorr := nivelcorr –1;

	\item  busca(TS,s,ref,achou);
Se não achou ou (achou e categoria $<>$ tipo) então erro(‘tipo não definido’)
Senão Se permissão\_tipo\_array então End\_tipo:= ref
 Senão Atualiza os campos nbytes e Pont\_tipo\_elementar da entrada do array 
 
 	\item  Se permissão\_tipo\_array então
 insere na TS as informações sobre o tipo de dados array;
 permissão\_tipo\_array := false;
 End\_tipo := Ponteiro do descritor do array
 Senão erro(‘array de várias dimensões não é permitido’)
 
 	\item permissão\_tipo\_array := true;
 pos := ref;
 
 	\item classet := valor;
    
    \item classet := referencia;
    
    \item Se declarado(TS, s, nivelcorr) então erro(‘id já declarado’);
insere(TS,s,ref); Atualiza os campos nível, categoria e classe\_transferência através da rotina seta\_atributos com:
nivelcorr, parâmetro, classet;
npar:= npar + 1;

	\item Atualiza o campo tipo\_p com ref (endereço do tipo elementar)
    \begin{enumerate}
    \item tipo\_p := ref;
    \end{enumerate}
    
    \item Atualiza o campo npar1 ou npar2 (do procedimento ou função que está apontado por pos);
npar:= 0;

	\item  Busca(TS, s, ref, achou);
Se não achou então erro(‘id não declarado’)
Senão obtém os atributos categoria e ref; c\_id:= categoria e p\_id := ref

	\item Se c\_id $<>$ função então erro (‘função não definida’) \{com abre-parênteses\}
    \begin{enumerate}
    \item  Se (c\_id $<>$ função ou c\_id $<>$ parâmetro ou c\_id $<>$ variável ou c\_id $<>$ constante) então erro(‘função, 			variável, parâmetro ou constante não definidos’) \{sem abre-parênteses\}
    \end{enumerate}
    
    \item  npar := npar + 1;
    
    \item Se c\_id $<>$ procedimento então erro(‘procedimento não definido’) \{com abre-parênteses\}
    
    \begin{enumerate}
    \item Se (c\_id $<>$ função ou c\_id $<>$ variável ou c\_id $<>$ parâmetro ou c\_id $<>$ procedimento) então
erro(‘função, procedimento, variável, ou parâmetro não definidos’) \{sem abre-parênteses\}
    \end{enumerate}
    
    \item Transformar a cadeia de dígitos em um tipo numérico (inteiro ou real, quando for o caso)
    
    \item Se a categoria do objeto apontado por p_id = procedimento
 então Se o número de parâmetros (npar1) <> npar então erro(‘incompatibilidade no número de parâmetros)
 senão Se o número de parâmetros (npar2) <> npar então erro(‘incompatibilidade no número de parâmetros);
npar:= 0;
    \end{enumerate}
    
    \subsection{Linguagem em EBNF, com checagens}
\begin{rules}[Programa e Bloco (EBNF)]
    \begin{numberedgrammar}

      <Program> ::= <Header> [<DeclSec>] <Block> \lit{.}

      <Header> ::=  \lit{program} identifier \lit{;}

      <Block> ::= \lit{begin} [<Statements>] \lit{end}

      <DeclSec> ::=  types <TypeDecls> ( <VarDeclSec><SubProgramDecls>
      \alt<SubProgramDecls>
      \alt <VarDeclSec> 
      \alt $\lambda$)
      \alt [<VarDeclSec>] <SubProgramDecls> 

      <TypeDecls> ::= \{<TypeDecl>\}

      <TypeDecl> ::= identifier rs(2) <TypeSpecification> rs(3)

      <TypeSpecification> ::= <ArraySpecification>
      \alt <RecordSpecification>

      <ArraySpecification> ::= array rs(11) <DataType> \lit{[}numericliteral\lit{]} \lit{;} 

      <RecordSpecification> ::= record <VarDecls> end;

      <VarDeclSec> ::= variables <VarDecls>

      <VarDecls> ::= \{<VarDecls>\} <VarDecl>

      <VarDecl> ::= <DataType> rs(5.a) <IdList> rs(5)\lit{;}

      <DataType> ::= real
      \alt integer
      \alt identifier
      \alt string 

      <IdList> ::=  \{identifier rs(4) \lit{,}\} identifier rs(4)

      <SubProgramDecls> ::= [<SubProgramDecls>]<SubProgramDecl>

      <SubProgramDecl> ::= <ProcDecl>
      \alt <FunctionDecl>

      <ProcDecl> ::= <ProcHeader> <SubProgramDeclSec> <Block> rs(9)\lit{;}

      <ProcHeader> ::= procedure identifier rs(6) rs(12)\lit{;}

      <SubProgramDeclSec> ::= [<ParamDeclSec>] <DeclSec> 
    \end{numberedgrammar}    
    \end{rules}
    
    \begin{rules}[Declarações  (EBNF)]
    \begin{numberedgrammar}
      <ParamDeclSec> ::= parameters <ParamDecls> rs(17)

      <ParamDecls> ::= [<ParamDecls>]<ParamDecl>

      <ParamDecl> ::= rs(13) [var rs(14)]<DataType> rs(16.a) identifier rs(8) rs(15) rs(16)\lit{;}

      <FunctionDecl> ::= <FunctionHeader> <SubProgramDeclSec> \\<FunctionBlock> rs(9) \lit{;}

      <FunctionHeader> ::= function identifier rs(6) rs(12) returns <Datatype> \lit{;}

      <FunctionBlock> ::= begin [<Statements>] <ReturnStatement> end

      <Statements> ::= \{ <Statements> \lit{;}\} <Statement>

      <Statement> ::= (read | write) <Variable> rs(18)
      \alt set <Variable> rs(1) rs(21.a) = <Expression> rs(1.a) rs(16.a)
      \alt if <Conditional\_Exp> then [<Statements>] <ElseClause>
      \alt while <Conditional\_Exp> do [<Statements>] endwhile
      \alt until <Conditional\_Exp> do [<Statements>] enduntil
      \alt call identifier rs(18) rs(21.a) [ rs(21) <ArgList> rs(23)]
      \alt <Ternary>

      <Ternary> ::= <Conditional\_Exp>?[<Statements>]:[<Statements>]

      <ElseClause> ::= [else] [<Statements>] endif

      <ReturnStatement> ::= \lit{;} return \lit{(} <Conditional\_Exp> \lit{)}

      <ArgList> ::= \lit{(} <Arguments> \lit{)}

      <Arguments> ::= \{<Conditional\_Exp> rs(20)\lit{,}\} <Conditional\_Exp> rs(20)
	  
      <Conditional\_Exp> ::= <Condition> [ ? <Conditional\_Exp>\\ : <Conditional\_Exp>] 
      
      <Condition> ::= \{<CompoundCondition> $\backslash$ \}    <CompoundCondition>

      <CompoundCondition> ::= \{<SimpleCondition> 
      \\ '\&'\} <SimpleCondition>

      <SimpleCondition> ::= \{<Expression> <RelOp>\} <Expression>

      <Expression> ::= \{<Term> <AddOp>\} <Term>


      <Term> ::= <Unary> \{<MultOp> <Factor>\}
      
    \end{numberedgrammar}\end{rules}
    
  \begin{rules}[Expressões (EBNF)]
    \begin{numberedgrammar}
      
      <Unary> ::= \{(\lit{+} | \lit{-})\} <Factor> 

      <Factor> ::= [ \textasciitilde ] (<Variable> rs(1.b)
      \alt numericliteral rs(1.c) rs(22)
      \alt stringliteral
      \alt \lit{(}<Conditional\_Exp>\lit{)}
      \alt identifier <Arglist>
      \alt <Ternary>)

      <Variable> ::= identifier rs(18)[\lit{[} <Expression> \lit{]}]\\ \{.identifier [\lit{[} <Expression> \lit{]}]\}

      <RelOp> ::= \lit{=}
      \alt \lit{!}
      \alt `>'
      \alt `<'
      \alt `<='
      \alt `>='

      <AddOp> ::= \lit{+}
      \alt \lit{-}

      <MultOp> ::= \lit{*}
      
    \end{numberedgrammar}
    \end{rules}
\section{Tipo Abstrato de Dados Escolhido}
	O Tipo Abstrato de Dados escolhido pelo grupo foi a lista encadeada não ordenada, por possuir alta eficiência na inserção (operação bastante frequente na análise semântica), com O(1). Uma eficiência razoável na busca, O(n), no pior caso. E uma eficiência também satisfatória na remoção, O(n), no pior caso. E, em termos de consumo de memória, essa estrutura também é econômica.
    
    Além disso, essa estrutura de dados foi escolhida pela facilidade de se trabalhar com ela, aumentando a produtividade do grupo, que já era familiarizado com ela. Assim, o uso da lista encadeada facilitou a criação do analisador semântico, armazenando a tabela de símbolos de maneira clara e intuitiva.
    
    Por fim, para parâmetros de \textit{procedures} e \textit{functions}, usou-se uma outra lista encadeada para salvá-los, na ordem em que estes são declarados.
    

\section{Utilzando e construindo a aplicação}
\subsection{Rodando o Parser}
Utilize o arquivo \emph{.jar} fornecido, ou construído nas subseções seguintes, da seguinte forma:
\begin{lstlisting}[language=bash,caption={Executando o Parser}]
#!/bin/bash
java -jar nomeDoJar.jar
\end{lstlisting}
O programa espera como entrada um caminho para um arquivo ou caso não haja argumento de entrada, este rodará iterativamente pelo \emph{STDIN} (\emph{standart input}). Como definido no projeto, caso aconteça algum erro, o parser parará imediatamente e informará qual era \emph{Token} esperada e qual foi a \emph{Token} recebida.

\subsection{Construindo o Parser}
Pode-se construir o \emph{Parser} através dos códigos fontes. Para isto algumas dependências são necessárias:
\begin{itemize}
\item Certifique-se de ter o Java Development Kit 8 ou superior instalado. Caso não o tenha, \href{https://docs.oracle.com/javase/8/docs/technotes/guides/install/install\_overview.html}{siga estes passos} ou utilze seu gerenciador de pacotes.
\item Instale o Gradle para automatização da \emph{build}. Encontre os passos para \href{https://gradle.org/install/}{sua distribuição aqui}. Alternativamente, use seu gerenciador de pacotes.
\end{itemize}
Descompacte os arquivos fontes do zip ou clone o repositório:
\begin{lstlisting}[language=bash,caption={Obtendo os arquivos fontes}]
#!/bin/bash
git clone https://github.com/abbudao/SCC0605 
\end{lstlisting}
Na pasta raiz do projeto, construa uma \emph{build}:
\begin{lstlisting}[language=bash,caption={Construção automatizada}]
#!/bin/bash
gradle build
\end{lstlisting}
Os arquivos gerados pelo \emph{JavaCC} estarão no caminho \emph{``build/generated''}
 e o jar automaticamente gerado no caminho \emph{``build/libs''}. Para a construção a suíte de testes já foi rodada e um report detalhado pode ser encontrado em \emph{``build/reports''}.
 
\subsection{Testes}
Para executar-se os testes, o segundo comando pode ser usado:
\begin{lstlisting}[language=bash,caption={Rodando a suíte de testes}]
#!/bin/bash
gradle test 
\end{lstlisting}
O resultado da suite será mostrado no \emph{STDOUT} e um resultado detalhado também pode ser encontrado
em \emph{``build/reports''}.



\end{document}





